% document vide aux normes de l'école pour le mémoire

% PREAMBULE

%package obligatoire : type de document
\documentclass[a4paper,12pt,twoside]{book}
% encodage
\usepackage[utf8]{inputenc}
\usepackage{fontspec}

% le package hyperref avec des options
\usepackage[pdfusetitle, pdfsubject ={Mémoire TNAH}, pdfkeywords={les mots-clés}]{hyperref}

%il faut mettre au moins une langue
\usepackage[english,french]{babel}

% configurer le document selon les normes de l'école
\usepackage[margin=2.5cm]{geometry} %marges
\usepackage{setspace} % espacement qui permet ensuite de définir un interligne
\onehalfspacing % interligne de 1.5
\setlength\parindent{1cm} % indentation des paragraphes à 1 cm

\usepackage{lettrine} % lettrines (pas obligatoire)


% bibliographie, ce qu'on verra à la prochaine séance
%\usepackage[backend=biber, sorting=nyt, style=enc]{biblatex}
%\addbibresource{biblio.bib}
%\nocite{*}

%si index, package pour index + makeindex

% + toutes la liste des packages nécessaires à votre document (si images, tableaux, schémas, etc.)

\author{Prénom Nom - M2 TNAH}
\title{Titre du mémoire}

% DOCUMENT
\begin{document}
	\frontmatter
	\begin{titlepage}
		\begin{center}
			
			\bigskip
			
			\begin{large}
				\'ECOLE NATIONALE DES CHARTES
			\end{large}
			\begin{center}\rule{2cm}{0.02cm}\end{center}
			
			\bigskip
			\bigskip
			\bigskip
			\begin{Large}
				\textbf{Prénom Nom}\\
			\end{Large}
		%selon le cas
			\begin{normalsize} \textit{licencié.e ès lettres}\\
				\textit{diplômé.e de master}
			\end{normalsize}
			
			\bigskip
			\bigskip
			\bigskip
			
			\begin{Huge}
				\textbf{TITRE DU MÉMOIRE}\\
			\end{Huge}
			\bigskip
			\bigskip
			\begin{LARGE}
				\textbf{SOUS-TITRE DU MÉMOIRE}\\
			\end{LARGE}
			
			\bigskip
			\bigskip
			\bigskip
			\begin{large}
			\end{large}
			\vfill
			
			\begin{large}
				Mémoire 
				pour le diplôme de master \\
				\og{} Technologies numériques appliquées à l'histoire \fg{} \\
				\bigskip
				2019
			\end{large}
			
		\end{center}
	\end{titlepage}
	
	\thispagestyle{empty}	
	\cleardoublepage
	
	\chapter*{Résumé}
	\addcontentsline{toc}{chapter}{Résumé}
	\medskip
	Résumé du mémoire en français. Cette page ne doit pas dépasser une page.\\
	
	\textbf{Mots-clés~:} une liste de mots-clés~; séparés par des points-virgules.
	
	\textbf{Informations bibliographiques~:} Prénom Nom, \textit{Titre du mémoire. Sous-titre du mémoire}, mémoire de master \og{}Technologies numériques appliquées à l'histoire\fg{}, dir. [Noms des directeurs], École nationale des chartes, 2019.
	
	\chapter*{Remerciements}
	\addcontentsline{toc}{chapter}{Remerciements}
	
	\lettrine{M}es remerciements vont tout d'abord à\dots
	
	%bibliographie ici
	%\printbibliography
	
	\chapter*{Introduction}
	\addcontentsline{toc}{chapter}{Introduction}
	
	\thispagestyle{empty}
	\cleardoublepage
	
	\mainmatter
	
	% là, le corps du mémoire, généralement trois parties
	\part{Une partie}
	
	\chapter{Un chapitre}
	\chapter{Un autre chapitre}
	
	\part{Une autre partie}
	
	%etc.
	
	
	\chapter*{Conclusion}
	\addcontentsline{toc}{chapter}{Conclusion}
	
	\appendix
	\part*{Annexes}	
	\addcontentsline{toc}{part}{Annexes}
	
	\backmatter
% index à mettre ici si index	
%	\printindex

% si figures
%	\listoffigures

	\tableofcontents
	
\end{document}